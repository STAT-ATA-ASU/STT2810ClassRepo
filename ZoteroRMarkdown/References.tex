\documentclass{article}\usepackage[]{graphicx}\usepackage[]{color}
%% maxwidth is the original width if it is less than linewidth
%% otherwise use linewidth (to make sure the graphics do not exceed the margin)
\makeatletter
\def\maxwidth{ %
  \ifdim\Gin@nat@width>\linewidth
    \linewidth
  \else
    \Gin@nat@width
  \fi
}
\makeatother

\definecolor{fgcolor}{rgb}{0.345, 0.345, 0.345}
\newcommand{\hlnum}[1]{\textcolor[rgb]{0.686,0.059,0.569}{#1}}%
\newcommand{\hlstr}[1]{\textcolor[rgb]{0.192,0.494,0.8}{#1}}%
\newcommand{\hlcom}[1]{\textcolor[rgb]{0.678,0.584,0.686}{\textit{#1}}}%
\newcommand{\hlopt}[1]{\textcolor[rgb]{0,0,0}{#1}}%
\newcommand{\hlstd}[1]{\textcolor[rgb]{0.345,0.345,0.345}{#1}}%
\newcommand{\hlkwa}[1]{\textcolor[rgb]{0.161,0.373,0.58}{\textbf{#1}}}%
\newcommand{\hlkwb}[1]{\textcolor[rgb]{0.69,0.353,0.396}{#1}}%
\newcommand{\hlkwc}[1]{\textcolor[rgb]{0.333,0.667,0.333}{#1}}%
\newcommand{\hlkwd}[1]{\textcolor[rgb]{0.737,0.353,0.396}{\textbf{#1}}}%

\usepackage{framed}
\makeatletter
\newenvironment{kframe}{%
 \def\at@end@of@kframe{}%
 \ifinner\ifhmode%
  \def\at@end@of@kframe{\end{minipage}}%
  \begin{minipage}{\columnwidth}%
 \fi\fi%
 \def\FrameCommand##1{\hskip\@totalleftmargin \hskip-\fboxsep
 \colorbox{shadecolor}{##1}\hskip-\fboxsep
     % There is no \\@totalrightmargin, so:
     \hskip-\linewidth \hskip-\@totalleftmargin \hskip\columnwidth}%
 \MakeFramed {\advance\hsize-\width
   \@totalleftmargin\z@ \linewidth\hsize
   \@setminipage}}%
 {\par\unskip\endMakeFramed%
 \at@end@of@kframe}
\makeatother

\definecolor{shadecolor}{rgb}{.97, .97, .97}
\definecolor{messagecolor}{rgb}{0, 0, 0}
\definecolor{warningcolor}{rgb}{1, 0, 1}
\definecolor{errorcolor}{rgb}{1, 0, 0}
\newenvironment{knitrout}{}{} % an empty environment to be redefined in TeX

\usepackage{alltt}
\usepackage[margin=1in]{geometry}
\usepackage[utf8]{inputenc} 
\usepackage{amsmath}
\usepackage{enumerate}
\usepackage[round]{natbib}
\usepackage[colorlinks=true, linkcolor=blue, citecolor=blue, urlcolor=blue, linktocpage=true, breaklinks=true]{hyperref}
\IfFileExists{upquote.sty}{\usepackage{upquote}}{}
\begin{document}

\title{Using \texttt{\textcolor{red}{z}otero} with \texttt{knitr}}
\author{Alan T. Arnholt}
\date{Spring 2015}
\maketitle




The only references from your \verb|Items.bib| file that will appear at the end of a document are those that have been cited in the text.  You can use \verb|nocite| to get a full bibliography but we will not discuss that further here.  You can use the following template to create your \verb|*.Rnw| file.
\begin{knitrout}
\definecolor{shadecolor}{rgb}{0.969, 0.969, 0.969}\color{fgcolor}\begin{kframe}
\begin{alltt}
\textbackslash{}documentclass\{article\}
\textbackslash{}usepackage[margin=1in]\{geometry\}
\textbackslash{}usepackage[utf8]\{inputenc\} 
\textbackslash{}usepackage\{amsmath\}
\textbackslash{}usepackage\{enumerate\}
\textbackslash{}usepackage[round]\{natbib\}
\textbackslash{}usepackage[colorlinks=true, linkcolor=blue, citecolor=blue, 
            urlcolor=blue, linktocpage=true, breaklinks=true]\{hyperref\}


\textbackslash{}begin\{document\}
\textbackslash{}title\{Your Title Here\}
\textbackslash{}author\{Your Name Here\}
\textbackslash{}maketitle

Whatever you have to say...say it here.

\textbackslash{}bibliographystyle\{chicago\}

\textbackslash{}bibliography\{Items\}
\textbackslash{}end\{document\}
\end{alltt}
\end{kframe}
\end{knitrout}

\noindent
To create an \verb|Items.bib|, 

\begin{itemize}
\item First, highlight the titles you want to select in \href{https://www.zotero.org}{\texttt{\textcolor{red}{z}otero}}.  
\item Second, for Windows users, right click on the highlighted items; for Mac users, Control-click on the highlighted items. 
\item Third, select \textbf{Export Items}.  Use the drop down menu to select \textbf{Bib\TeX} not \textbf{Bib\LaTeX} as the format. 
\item Fourth, click \verb|OK|.  Change the name of the file to \verb|Items.bib| in the \textbf{Save As:} box. 
\item Fifth, click \textbf{Save}.
\end{itemize}

For examples of how to cite articles with \texttt{natbib}, see the reference sheet
\href{http://ftp.math.purdue.edu/mirrors/ctan.org/macros/latex/contrib/natbib/natnotes.pdf}{natnotes.pdf}.
I can really talk according to \citet{beck2014} and \citet{dean2014}.
The mean is 28 for YUMMIES \citep{murp2012}. \citet{rich2013} defines a YUMMIE as a GIDGO.

\nocite{*}
\bibliographystyle{chicago}  % or chicago I like jss
\bibliography{Items,Rpkgs}
\end{document}
